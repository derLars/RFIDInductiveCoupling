%----------------------------------------------------------------------------------------
%	RFID
%----------------------------------------------------------------------------------------
\newcommand*{\rfid}{\begingroup
\section{RFID}\label{sec:rfid}
{\noindent RFID steht für Radio Frequency Identification und beschreibt eine Technologie der berührungslosen Übertragung von Informationen. Die beiden Grundlegenden komponenten eines RFID-Systems sind ein Lesegerät und ein Transponder. Das Lesegerät, oftmals stationär eingesetzt, wartet darauf dass ein Transponder in das Lesegebiet eintritt. Ist dies der Fall beginnen das Lesegerät und der Transponder zu kommunizieren. Der Transponder selbst kann die unterschiedlichsten Bauformen annehmen und ist dadurch sehr variabel für viele Gebiete einsetzbar. Kleinste Bauformen finden sich unter Anderem in der Tieridentifikation wieder, wo der Transponder die Form eines kleinen Röhrchens von wenigen Millimetern Umfang und Länge annimmt und so fast komplett ungemerkt Platz unter der Haut des Tieres findet. Üblich sind auch Transponder in Form einer Scheckkarte nach ISO/IEC 7810, sodass diese in jede übliche Brieftasche passen und so ein weites Feld an Einsatzgebieten wie elektronische Fahrkarten, Mitarbeiterausweise oder Konzerttickets einnehmen. Neben der Bauform klassifiziert noch die maximale Lesereichweite ein RFID-System. Diese Eigenschaft ist stark davon abhängig ob es sich um ein aktives oder ein passives Transpondersystem handelt. Aktive Transponder benötigen eine eigene Energiequelle um zum Leben erweckt zu werden. Dies ermöglicht in der Regel eine sehr hohe Reichweite, bedeutet aber auch eine größere Bauform des Transponders. Passive Transponder hingegen beziehen die Energie komplett von dem Lesegerät. Die mögliche Distanz zu dem Lesegerät ist dadurch sehr begrenzt und kann von wenigen Zentimetern bis hin zu ein paar Metern reichen. Erreichbare Datenraten sind hauptsächlich abhängig von der verwendeten Funkfrequenz des RFID-Systems\footnote{Finkenzeller, Klaus: RFID Handbuch. 5. Auflage: Carl Hanser Verlag München, S. 23 f.}.

\frequencies

}
\endgroup}
%----------------------------------------------------------------------------------------


%----------------------------------------------------------------------------------------
%	Frequencies
%----------------------------------------------------------------------------------------
\newcommand*{\frequencies}{\begingroup
\subsection{Frequenzen}\label{sec:frequencies}
{\noindent Eine Anlage zu betreiben, welche elektromagnetische Wellen abstrahlt, unterliegt gewissen Richtlinien. Deshalb ist es nicht ohne Weiteres möglich im Betrieb eines RFID-Systems eine beliebige Frequenz zu verwenden, da die Rechtevergabe der Frequenzen dem jeweiligen Staat bzw. durch Zusammenschlüsse geschaffene Organisationen obliegt. In Europa beispielsweise koordiniert die Zuteilung der Frequenzen fast aller europäischer Länder die CEPT (Conférence Européenne des Postes et Télécommunications) bzw. die daraus entstandene Organisation ETSI (European Telecommunications Standards Institute). Möchte man ein RFID-System innerhalb erlaubter Richtlinien betreiben, bieten sich Langwellen-, ISM- und SRD-Frequenzen an\footnote{Finkenzeller: RFID Handbuch, S. 173 ff.}. 

\lfFrequencies
\ismFrequencies
\srdFrequencies

}
\endgroup}
%----------------------------------------------------------------------------------------

%----------------------------------------------------------------------------------------
%	SUB: LF
%----------------------------------------------------------------------------------------
\newcommand*{\lfFrequencies}{\begingroup
\subsubsection{Langwellen-Frequenzen}\label{sec:lffrequencies}
{\noindent Die Langwellen-Frequenzen im Bereich von 9 bis 135KHz stehen tendenziell für die Verwendung mit einem RFID-System zur Verfügung. Da dieser Bereich allerdings auch von anderen Diensten, wie der Schiffsnavigation oder dem Zeitzeichenfunkdienst, verwendet wird, dürfen bestimmte Frequenzen nur mit einer geringeren maximalen Feldstärke betrieben werden. Ein Beispiel hierfür ist die Frequenz 77,5KHz. Über Diese empfangen Funkuhren innerhalb Mitteleuropa die Information zur aktuellen Uhrzeit\footnote{Finkenzeller: RFID Handbuch, S. 175}. In RFID-Systemen findet die Frequenz 125KHz häufig Verwendung und wird unter Anderem in der Fernverriegelung von Autos eingesetzt.
}
\endgroup}
%----------------------------------------------------------------------------------------

%----------------------------------------------------------------------------------------
%	SUB: ISM
%----------------------------------------------------------------------------------------
\newcommand*{\ismFrequencies}{\begingroup
\subsubsection{ISM-Frequenzen}\label{sec:ismfrequencies}
{\noindent ISM-Frequenzen (Industrial Scientific and Medical) sind international gebührenfrei verfügbare Frequenzen. Beispiele Hierfür sind unter Anderem 13,56MHz, welche eine häufige Verwendung für RFID-Systeme findet und 2,45GHz welche unter Anderem auch als WLAN-Frequenz und in einem Mikrowellenherd eingesetzt wird.
}
\endgroup}
%----------------------------------------------------------------------------------------

%----------------------------------------------------------------------------------------
%	SUB: SRD
%----------------------------------------------------------------------------------------
\newcommand*{\srdFrequencies}{\begingroup
\subsubsection{SRD-Frequenzen}\label{sec:srdfrequencies}
{\noindent SRD-Frequenzen (Short Range devices) sind für den professionellen und privaten Einsatz im Bereich Kurzstreckenfunk vorgesehen. Ein Vertreter der SRD-Frequenzen ist 868MHz, welche häufig im Bereich von Funkgesteuerten Steckdosen eingesetzt wird.
}
\endgroup}
%----------------------------------------------------------------------------------------
